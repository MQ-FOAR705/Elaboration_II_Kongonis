\documentclass{article}
\usepackage[utf8]{inputenc}

\title{Elaboration II}
\author{Thomas Kongonis 44618468}
\date{September 2019}

\begin{document}

\maketitle
\tableofcontents

\section{Reassertion of the Problem}

\begin{itemize}

\item{\textbf{Nature of problem:} The problem that is being posed throughout this process is specifically a textual
analysis problem.}

\begin{itemize}

\item{\textbf{Macro Process:} The process that is being posed for analysis and elaboration is related to the difference within multiple translations of eastern philosophical texts.}

\item{\textbf{Context:} This concerns specifically east Asian languages for the scope of this proof of concept. The key text that will be utilised as the exemplar of this process will be the Laozi or Dao de Jing as it is also called.}

\end{itemize}

\end{itemize}


\section{Breakdown of Processes}

\begin{itemize}

\item{\textbf{First Step- Finding and Reading of Text:} The Laozi has 81 chapters, for this example chapter one will be utilised after using the internet or library to find a translation.}
\item{\textbf{Second Step-Diagnosing Translation Problem:} The opening sentence of this text was taken from an open source website 'ctext.org/dae-de-jing/'. This translation opens with the quote "\textit{The Dao that can be trodden is not the enduring and unchanging Dao.}"}

\item{\textbf{Third Step-Consulting Secondary Translation} In this stage i would consult a secondary translation, this is because the first translation is a bit clunky and quite confusing. Another permutation of words is necessary for my understanding. for the sake of this example i will utilise the most common translation of this sentence, which is "\textit{The Dao that can be named is not the eternal Dao}."}
\item{\textbf{Fourth Step-Comparison:} In this step, the key points of difference would be diagnosed. In this case we have the words "trodden" and "named" alongside "enduring and unchanging" and "eternal". These appear quite different in meaning even though there is a common thread.}
\item{\textbf{Fifth Step-Draw Conclusions:} In this step, these two words could also be compared with other synonyms out of context of the text for greater understanding. This would normally be done on pen and paper.}
\item{\textbf{Sixth Step- Repeat Until Clarity is Reached:} The issue with this approach is that if i was not happy with my understanding of these translations, then i would have to repeat and continue finding new translations. This is a significant pain point and can become increasingly tedious.}


\end{itemize}



\section{Concept}

\begin{itemize}

\item{\textbf{Converting Pains to Gains:} This entire process listed above is not one that can be totally avoided, however it can be automated and made significantly easier.}

\begin{itemize}

\item{\textbf{Preliminary Revision:} The original idea for the proof of concept, was to create a way to translate and highlight single words among a database of 4-5 public domain translations of the Dao de Jing. Upon consultation, the conclusion was made that this would be unfeasible due to the nature of what would be required in relation to my skill level.}

\item{As such, the project was altered at this stage to imagine a way in which this database of texts could be specified and placed alongside each other down to the chapter level rather than the word level. This doesn't totally automate the process but does create a significant gain in what was previously a long and drawn out process.}


\end{itemize}

\end{itemize}

\section{Applicable technology}

\begin{itemize}

\item{\textbf{Textual Analysis:} The initial elaboration of the project included textual analysis software. Tools such as Nvivo and Voyant were listed. After the testing of these tools initially, it was discovered that they could not do what i required them to do for this project. As such, i needed to readjust.}

\item{\textbf{The Unix Shell} Upon learning about the shell and seeking consultation, the power of the shell came up as a tool that could be utilised as a way to comb a database of texts and pull up the relevant chapters. However, this requires files of a particular nature that are tagged with titles.}

\end{itemize}

\section{Testing and Revising}

\begin{itemize}
\item{\textbf{Searching for Digital library and api:} On suggestion, i begun to search for a digital library that would contain all of the translations i wanted archived. Extended research has come up with no such repositories like that of the Perseus Digital Library. With that being said, i have found at least 3 public domain translations in various places and am confident with at least finding 5.}

\item{It appears likely that i will need to download these translations as text documents or turn them into text documents. From this I will need to convert them into a format in which I can tag the chapters and make them compatible with running a shell code}

\item{This will essentially make the project comprise 3 parts. Finding my translations in the first step, converting them into a usable format in the second step and finally writing a shell script to show the relevant chapters of each of the translations side by side as I require them.}



\end{itemize}





\section{Results}


\begin{itemize}

\item{I believe that the findings of this elaborated research into the issue have not only framed the problem correctly but have also created a potential proof of concept project that is challenging enough for my skill level.}

\item{After converting the texts, putting them in a directory and running a shell script on them that I have written, the original processed will be changed to simply me typing one line of code into the shell and pulling up the information I need alongside each other so I can see the different translations. This tool will most likely save me hours.}



\end{itemize}







\end{document}
